%%% 中文版式

% 中文排版惯例,段首缩进两个字符
\usepackage{indentfirst}
% 尺寸设置
\setlength{\parindent}{2em}
\linespread{1.4} % 汉字显示不至于太密集
\addtolength{\parskip}{2ex}
\addtolength{\voffset}{-0.2in}
\addtolength{\textheight}{0.8in}
\addtolength{\footskip}{10pt}
%\addtolength{\textwidth}{66pt} % 中文版幅不宜过窄

% 个人认为有些中文字体标点符号左右宽度过大
\xeCJKDeclarePunctStyle {mine}
{
  fixed-punct-ratio=nan,
  fixed-margin-ratio=0.5,
  mixed-margin-ratio=0.75,
}
\xeCJKsetup{PunctStyle=mine}
\xeCJKsetwidth{ ( ) }{0.6em}
\xeCJKsetkern{”}{(}{0.3em}
\xeCJKsetkern{)}{。}{0.3em}
\xeCJKsetkern{)}{,}{0.3em}
\CJKsetecglue{\,}

% 标题格式
\usepackage{titlesec}
% \titleformat{\section}[block]{\Large\bf}{\S \thesection}{1em}{}
\titleformat*{\paragraph}{\itshape}
\titlespacing*{\paragraph} {\parindent}{0.2em}{\baselineskip}
% \renewcommand{\chaptername}{第\CJKnumber{\thechapter}章}
% \newcommand{\sectionname}{节}

\renewcommand{\figurename}{图}
\renewcommand{\tablename}{表}

\usepackage[title]{appendix}
\newcommand{\sectionbreak}{\clearpage}
\renewcommand{\appendixname}{附录}
% \renewcommand{\bibname}{参考文献}
\renewcommand{\contentsname}{目~录}
\renewcommand{\listfigurename}{图~目~录}
\renewcommand{\listtablename}{表~目~录}
\renewcommand{\indexname}{索~引}
\renewcommand{\abstractname}{\Large{摘~要}}
% \nonfrenchspacing
% \newcommand{\zhem}[1]{\ifnum\spacefactor>1900 {\em #1}\kern0.2em\else \kern0.2em{\em #1}\kern0.2em\fi}
\newtheorem{theorem}{定理}[section] % Theorem
\newtheorem{lemma}[theorem]{引理} % Lemma
\newcounter{ax}[section] % Axiom
\newtheorem{axiom}[ax]{公设}
\newtheorem{prop}[theorem]{命题} % Proposition
\newtheorem{coro}[theorem]{推论} % Corollary
\newcommand{\zhem}[1]{\kern0.1em{\em {#1}}\kern0.1em}
\newcommand{\subjectto}{\ \ s.t.\;}
\renewcommand{\today}{\number\year 年 \number\month 月 \number\day 日}

%%% TODO(luke): confict with \arc macro defined in preamble_maths.tex
% \usepackage{diagbox} % 中式表格斜线表头
