\usepackage{IEEEtrantools}
\IEEEeqnarraydefcolsep{0}{\leftmargini}

\newcommand{\Eqref}[1]{\raisebox{0.12ex}{\eqref{#1}}}
\newcommand{\Thref}[1]{\raisebox{0.12ex}{\eqref{#1}}}

\makeatletter % `@' now normal "letter"
\@addtoreset{equation}{section}
\makeatother  % `@' is restored as "non-letter"
%\renewcommand\theequation{\oldstylenums{\thesection}%
%                   .\oldstylenums{\arabic{equation}}}
\renewcommand\theequation{{\thesection}-{\arabic{equation}}}

\usepackage{gensymb}
\usepackage{mathtools} % mathtools loads amsmath, use [fleqn] to flush all equations to left
\usepackage{amsthm}
\usepackage{amsfonts}
\usepackage{amssymb}
\usepackage{mathrsfs}
\usepackage{siunitx} % \ang{30} -> 30°
% \usepackage{fourier}  % don't use together with ``mathpazo'' !

\newcommand{\envert}[1]{\kern-2pt\left\lvert#1\right\rvert}
\let\abs=\envert
\makeatletter
\newcommand*\bigcdot{\mathpalette\bigcdot@{.5}}
\newcommand*\bigcdot@[2]{\mathbin{\vcenter{\hbox{\scalebox{#2}{$\m@th#1\bullet$}}}}}
\makeatother
%\newcommand{\subjectto}[0]{s.t.\enspace}
\newcommand{\Group}[2]{(#1,#2)}
\newcommand{\Ring}[2]{(R,#1,#2)}
\newcommand{\Field}[2]{(F,#1,#2)}

\mathchardef\mhyphen="2D

\ifx\zhtheorem\undefined
\theoremstyle{definition}
\newtheorem{definition}{Definition}[section]
\newtheorem{theorem}{Theorem}[section]
\newtheorem{lemma}{Lemma}[section] % Lemma
\newtheorem{notation}{Notation}[section]
%\newtheorem{proof}{Proof}[section] % proof
\fi

\newcommand{\arc}[1]{{%
  \setbox9=\hbox{#1}%
  \ooalign{\resizebox{\wd9}{\height}{\texttoptiebar{\phantom{A}}}\cr#1}}}

%\everymath{\displaystyle}

% 只要思想不滑坡,方法总比困难多!Unicode: x2225, ∥
\makeatletter
\newcommand{\slantparallel}{\mathrel{\mathpalette\new@parallel\relax}}
\newcommand{\new@parallel}[2]{%
  \begingroup
  \sbox\z@{$#1T$}% get the height of an uppercase letter
  \resizebox{!}{\ht\z@}{\raisebox{\depth}{$\m@th#1/\mkern-5mu/$}}%
  \endgroup
}
\makeatother
